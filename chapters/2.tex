% 调整章节标题与顶部间距
\quad \\
\vspace{-20mm}

\section{论文结构及文字格式}



\subsection{本章引言}

\subsection{论文结构}

学位论文包括前置部分、主体部分和结尾部分共三大部分\upcite{王兰芬2010Swarm},各部分组成及顺序如所示\upcite{谢希仁2006计算机网络教程}。

学位论文各部分独立为一部分,每部分应从新的一页开始\upcite{吕学勤2013求解}。

论文的正文(中间各章)是论文的核心部分\upcite{LSGAN, DCGAN, MR-GAN},一般由标题、文字叙述、图、表格和公式等部分构成。


\subsection{字数要求}

\subsubsection{硕士论文要求}

各学科和学部自定。

\subsubsection{博士论文要求}

各学科和学部自定。

\subsection{字体和段落}
学位论文中的中文统一用宋体,数字和英文统一用Times New Roman字体。从中文摘要开始,所有文字段落和标题行间距均取固定值20磅;所有段落按两端对齐、首行缩进2个全角字符方式书写内容。

中、英文混排时,除小数点以及引用的分图序号、公式序号等外,宜使用全角标点符号(逗号、冒号、括号、引号等);英文段落中,符号使用应遵循英文书写习惯,统一使用半角符号,并规范使用空格。

其他要求:

(1)各级标题不得置于页面的最后一行,即须与下段同页;

(2)两个标题之间无正文时,第二个标题的段前距设置为0磅;

(3)图、表、公式统一采用单倍行距;

(4)只有一、两行文字的,不得单独作为一页内容;除各章最后一页外,中间页面不得出现较大空白;

(5)必要时,可在规定的格式要求基础上适当微调,以利于排版,但显示效果不得与规定的格式要求存在明显差距。


\subsection{本章小结}
本章介绍了……




